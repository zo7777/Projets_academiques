% Options for packages loaded elsewhere
\PassOptionsToPackage{unicode}{hyperref}
\PassOptionsToPackage{hyphens}{url}
%
\documentclass[
]{article}
\usepackage{amsmath,amssymb}
\usepackage{lmodern}
\usepackage{iftex}
\ifPDFTeX
  \usepackage[T1]{fontenc}
  \usepackage[utf8]{inputenc}
  \usepackage{textcomp} % provide euro and other symbols
\else % if luatex or xetex
  \usepackage{unicode-math}
  \defaultfontfeatures{Scale=MatchLowercase}
  \defaultfontfeatures[\rmfamily]{Ligatures=TeX,Scale=1}
\fi
% Use upquote if available, for straight quotes in verbatim environments
\IfFileExists{upquote.sty}{\usepackage{upquote}}{}
\IfFileExists{microtype.sty}{% use microtype if available
  \usepackage[]{microtype}
  \UseMicrotypeSet[protrusion]{basicmath} % disable protrusion for tt fonts
}{}
\makeatletter
\@ifundefined{KOMAClassName}{% if non-KOMA class
  \IfFileExists{parskip.sty}{%
    \usepackage{parskip}
  }{% else
    \setlength{\parindent}{0pt}
    \setlength{\parskip}{6pt plus 2pt minus 1pt}}
}{% if KOMA class
  \KOMAoptions{parskip=half}}
\makeatother
\usepackage{xcolor}
\usepackage[margin=1in]{geometry}
\usepackage{graphicx}
\makeatletter
\def\maxwidth{\ifdim\Gin@nat@width>\linewidth\linewidth\else\Gin@nat@width\fi}
\def\maxheight{\ifdim\Gin@nat@height>\textheight\textheight\else\Gin@nat@height\fi}
\makeatother
% Scale images if necessary, so that they will not overflow the page
% margins by default, and it is still possible to overwrite the defaults
% using explicit options in \includegraphics[width, height, ...]{}
\setkeys{Gin}{width=\maxwidth,height=\maxheight,keepaspectratio}
% Set default figure placement to htbp
\makeatletter
\def\fps@figure{htbp}
\makeatother
\setlength{\emergencystretch}{3em} % prevent overfull lines
\providecommand{\tightlist}{%
  \setlength{\itemsep}{0pt}\setlength{\parskip}{0pt}}
\setcounter{secnumdepth}{-\maxdimen} % remove section numbering
\ifLuaTeX
  \usepackage{selnolig}  % disable illegal ligatures
\fi
\IfFileExists{bookmark.sty}{\usepackage{bookmark}}{\usepackage{hyperref}}
\IfFileExists{xurl.sty}{\usepackage{xurl}}{} % add URL line breaks if available
\urlstyle{same} % disable monospaced font for URLs
\hypersetup{
  pdftitle={NBA\_dashboard},
  pdfauthor={YEHOUENOU\_KWEKEU\_SY},
  hidelinks,
  pdfcreator={LaTeX via pandoc}}

\title{NBA\_dashboard}
\author{YEHOUENOU\_KWEKEU\_SY}
\date{2023-03-18}

\begin{document}
\maketitle

\hypertarget{i-intro}{%
\section{\texorpdfstring{\textbf{I) Intro}}{I) Intro}}\label{i-intro}}

\hypertarget{ii-contexte}{%
\section{\texorpdfstring{\textbf{II)
Contexte}}{II) Contexte}}\label{ii-contexte}}

\hypertarget{iii-base-de-donnees}{%
\section{\texorpdfstring{\textbf{III) Base de
Donnees}}{III) Base de Donnees}}\label{iii-base-de-donnees}}

Les donnees utilisees pour ce projet ont ete extraites du site:
\url{https://www.kaggle.com/datasets/nathanlauga/nba-games}. Ce lien
mets a notre disposition 5 tableaux de donnees en format csv nommes:

\begin{itemize}
\tightlist
\item
  games.csv: constitue de tous les matchs de la NBA de 2003 a Decembre
  2022 et des performances des equipes lors de ces matchs;
\item
  games\_details.csv: constitue des performances des joueurs pendant de
  chaque match de 2003 a Decembre 2022;
\item
  players.csv: constitue des informations des joueurs de la NBA de 2003
  a Decembre 2022;
\item
  ranking.csv: constitue des rangs des equipes apres chaque match(separe
  par conference Est et Ouest) de 2003 a Decembre 2022;
\item
  teams.csv: constitue des informations des equipes de la NBA.
\end{itemize}

Cependant, ces tableaux de donnees a l'etat brut ne pouvaient pas nous
permettre de parvenir a notre objectif. Pour cette raison, des
manipulations de ces tableaux de donnees etaient inevitables afin de
creer les tableaux finaux dont nous avions besoin.

\hypertarget{iv-traitement-de-donnees}{%
\section{\texorpdfstring{\textbf{IV) Traitement de
donnees}}{IV) Traitement de donnees}}\label{iv-traitement-de-donnees}}

Le traitement de donnees s'est fait en 2 partie. La premiere partie
concerne les joueurs de la NBA et la seconde concerne les equipes de la
NBA. Ce travail de traitement etait entierement centre sur la creation
de tableaux de donnees a partir de ceux importes.

\hypertarget{nettoyage-de-players}{%
\subsubsection{\texorpdfstring{\textbf{1) Nettoyage de
``players''}}{1) Nettoyage de ``players''}}\label{nettoyage-de-players}}

\hypertarget{traitement-de-donnees-concernant-les-equipes}{%
\subsubsection{\texorpdfstring{\textbf{2) Traitement de donnees
concernant les
equipes}}{2) Traitement de donnees concernant les equipes}}\label{traitement-de-donnees-concernant-les-equipes}}

Nous avons divise cette etape en 2 sous-etape selon nos besoins. En
effet, l'une des etapes concerne les statistiques de chaque equipe a
chaque saison; et l'autre concerne les rangs des equipes a la fin de
chaque saison.

\hypertarget{a-traitement-de-donnees-concernant-les-statisiques-des-equipes}{%
\subparagraph{\texorpdfstring{\textbf{a) Traitement de donnees
concernant les statisiques des
equipes}}{a) Traitement de donnees concernant les statisiques des equipes}}\label{a-traitement-de-donnees-concernant-les-statisiques-des-equipes}}

\hypertarget{difficultee-rencontrees-sur-la-visualisation-de-donnees}{%
\subsubsection{\texorpdfstring{\textbf{Difficultee rencontrees sur la
visualisation de
donnees}}{Difficultee rencontrees sur la visualisation de donnees}}\label{difficultee-rencontrees-sur-la-visualisation-de-donnees}}

\end{document}
